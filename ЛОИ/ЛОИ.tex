\documentclass[11pt]{article}
\usepackage[utf8]{inputenc}
\usepackage[russian]{babel}
\usepackage{amsmath}
\usepackage{ntheorem}
\usepackage{enumitem}
\usepackage{tabulary}
\usepackage[left=2cm, right=2cm, top=2cm]{geometry}

\newtheorem*{df}{Определение}
\newtheorem*{theorem}{Теорема}
\theoremstyle{break}
\newtheorem*{example}{Пример:}
\newtheorem*{proof}{Доказательство:}

\begin{document}
\raggedright
\begin{sloppypar}
\section*{Введение}

\begin{df}
Грамматика $G = <\Sigma>$
\end{df}

pos = init + delta * 60;
\begin{enumerate}
\item Лексический анализ \\
<id, 1> <=> <id, 2> <+> <id, 2> <*> <const> <;>
\item Синтаксический анализ (разложение в дерево)
\item Семантический анализ
\item Промежуточное представление \\
$t_1 = delta * 60$ \\
$t_2 = init + t_1$
\end{enumerate}

\section*{Иерархия Хомского}
малые латинские буквы - терминалы \\
большие - нетерминалы

\begin{tabulary}{\linewidth}{|C|c|C|c|}
\hline
Вид грамматики & Правила & Распознаватель & Класс языков \\
\hline\hline
Общего вида (неограниченные) & $\alpha \rightarrow \beta$ & МТ & RecEn \\
\hline
Контекстно-зависимые(КЗ) & $\alpha A \beta \rightarrow \alpha \gamma \beta$ & Линейный ограниченный автомат (LBA) & КЗЯ \\
\hline
Контекстно-свободные(КС) & $A \rightarrow \beta$ & Недетерминированный автомат с магазинной памятью (PDA) & КСЯ \\
\hline
Праволинейные & $A \rightarrow \gamma B$ & ДКА & Регулярные \\
\hline
\end{tabulary}

\begin{example}
$S \rightarrow ASB|\lambda$ \\
$AB \rightarrow BA$ \\
$A \rightarrow a$ \\
$B \rightarrow b$

Эквивалентная грамматика: \\
$S \rightarrow aB|bA$ \\
$A \rightarrow aS|bAA$ \\
$B \rightarrow bS|aBB$ \\
$A \rightarrow a$ \\
$B \rightarrow b$

Значит исходная грамматика - праволинейная.
\end{example}

Регулярные $\subset$ КСЯ $\subset$ КЗЯ $\subset$ RecC $\subset$ RecEn

\begin{df}
	Язык обладает св-м $P$, если $\exists$ грамматика со св-м $P$, его порождающая
\end{df}

\section*{КСГ и КСЯ}
\begin{df}
 Упорядоченное дерево - дерево с заданным линейным порядком со св-ми:
 \begin{enumerate}
  \item если $x$ - сын $y$, то $x \geq y$
  \item если $x$ и $y$ - братья и $x \leq y$, то для всех сыновей $z$ узла $x$: $z \leq y$
 \end{enumerate}
\end{df}

\begin{df}
 Дерево вывода цепочки w в грамматике G - упорядоченное дерево со св-ми:
 \begin{enumerate}
  \item Узлы - нетерминалы, корень - акисиома, листья - терминалы или $\lambda$, причем у листьев $\lambda$ нет братьев

  \item Если у узла $x$ сыновья $y_1 \leq ... \leq y_n$, то существует правило вывода $X \rightarrow Y_1...Y_n$

  \item Если все листья дерева имеют метки $a_1 \leq ... \leq a_n$, то $w = a_1...a_n$
 \end{enumerate}
\end{df}

\begin{df}
 Вывод цепочки w ($S \rightarrow \alpha_1 \rightarrow ... \rightarrow \alpha_n = w$) в G представлен деревом T, если существует набор стандартных поддеревьев $T_1 ... T_n$ такой, что упорядоченные листья $T_i$ являются $\alpha_i$
\end{df}

\begin{df}
 $T'$ - стандартное поддерево $T$, если:
 \begin{enumerate}
  \item Корни $T$ и $T'$ совпадают
  \item если узел лежит в $T'$, то он либо лист в $T'$, либо все его сыновья лежат в $T'$
 \end{enumerate}
\end{df}

Одной цепочке могут соответствовать несколько деревьев.

\begin{df}
 Грамматика однозначна, если любая цепочка имеет единственное дерево вывода. \\
 Язык однозначен, если существует порождающая его однозначная грамматика.
\end{df}

\section*{Праволинейная грамматика}
$A \rightarrow \alpha B$

$A \rightarrow \lambda$

\begin{theorem}
  Праволинейная грамматика порождает регулярный язык.
\end{theorem}
\begin{proof}
  $G = (\Sigma, \Gamma, P, S)$
  
  $A = (\Sigma, \Gamma, \delta, S, F)$
  
  $F = \{A \in \Gamma | (A \rightarrow \lambda) \in P\}$
  
  $\delta(A, a) = B \Leftrightarrow (A \rightarrow aB) \in P$
\end{proof}

\section*{Преобразования грамматик}
\begin{enumerate}
\item Приведенные грамматики.
  \begin{df}
    Нетерминал $A \in \Gamma$ - производящий, если из него можно получить терминальную цепочку.
  \end{df}
  \begin{df}
    Нетерминал $A \in \Gamma$ - достижимый, если $S \Rightarrow \alpha A \beta$
  \end{df}
  \begin{df}
    Грамматика - приведенная, если все ее нетерминалы достижимые и проиводящие.
  \end{df}
  Алгоритм нахождения $\Gamma_r$ - мн-ва производящих символов:
  $\Gamma \leftarrow S$
  
  $\Gamma_1 = \Gamma \cup \{A|(B \rightarrow \alpha A \beta) \in P, B \in \Gamma\}$
  
  Алгоритм нахождения $\Gamma_p$ - мн-ва достижимых символов:
  $\Gamma = \{A|(A \rightarrow w) \in P\}$
  $\Gamma_1 = \Gamma \cup \{A|(A \rightarrow \gamma) \in P, \gamma \in (\Sigma \cup \Gamma)\}$
  
  \begin{theorem}
    Для любой КСГ $G$ существует эквивалентная ей приведенная грамматика.
  \end{theorem}
  \begin{proof}
    $G = (\Sigma, \Gamma, P, S)$
    
    Находим $\Gamma_p$. Если $S \notin \Gamma_p$, то $G' = (\Sigma, \O, \O, \O)$. Иначе $\overset{\triangle}{\Gamma} = (\Sigma, \Gamma_p, p', S)$
    
    $\overset{\triangle}{P} = \{(A \rightarrow \gamma) \in P|A, \gamma \in (\Sigma \cup \Gamma_p)\}$
    
    Находим $(\Gamma_p)_r$. Все символы будут достижимые и производящие в $G'$. Порядок важен.
  \end{proof}
  \item $\lambda$-сведение грамматики.
  \begin{df}
    $A \in \Gamma$ - аннулирующий, если $A \Rightarrow \lambda$
  \end{df}
\end{enumerate}
  Алгоритм нахождения $Ann(G)$ - мн-ва аннулирующих нетерминалов.
  $Ann(G) = \{A \in \Gamma|(A \rightarrow \lambda) \in P\}$
  $Ann_1(G) = \{A \in \Gamma|(A \rightarrow \gamma) \in P, \gamma \in (Ann(G))*\}$
  \begin{df}
    $\lambda$ - свободная грамматика - грамматика, которая либо не содержит аннулирующих правил ($A \rightarrow \lambda$), либо содержит единственное такое правило из $S$ и $S$ не встречается в правых частях правил вывода.
  \end{df}
  
  \begin{theorem}
    Любая грамматика эквивалентна некоторой $\lambda$-свободной грамматике
  \end{theorem}
  \begin{proof}
    $G = (\Sigma, \Gamma, P, S)$
    
    Если $\lambda \in L(G)$, то $\Gamma' = \Gamma \cup S'$, $P' = P \cup \{(S' \rightarrow \lambda), (S' \rightarrow S)\}$. Иначе не изменяются.
    
    Построим $Ann(G)$.    
    
    $\beta \prec \gamma$, если $\beta$ - подпослед-ть $\gamma$ и все символы $\gamma \setminus \beta$ - аннулирующие.
    
    $P' = \{(A \rightarrow \beta)| (A \rightarrow \gamma) \in P, \beta \prec \gamma, \beta \neq \lambda\}$
    
    Проверим, что $L(G) = L(G')$.
    
    \begin{enumerate}
    \item $w \in L(G')$, $S \rightarrow \alpha_1 \rightarrow ... \rightarrow \alpha_n = w$
    
      Если $A \rightarrow_{G'} \beta$, то $A \rightarrow_G \gamma$, где $\beta \prec \gamma$
    
    \item $w \in L(G)$
    
      ...
    \end{enumerate}
  \end{proof}
\end{sloppypar}
\end{document}

