\documentclass[11pt]{article}
\usepackage[utf8]{inputenc}
\usepackage[russian]{babel}
\usepackage{amsmath}
\usepackage{ntheorem}
\usepackage{enumitem}
\usepackage{tabulary}
\usepackage[left=2cm, right=2cm, top=2cm]{geometry}
\usepackage[T1]{fontenc}

\newenvironment{me}{
  \begin{equation}
    \begin{gathered}
      }{
    \end{gathered}
  \end{equation}
}

\newenvironment{ce}{
  \begin{equation}
    \begin{cases}
      }{
    \end{cases}
  \end{equation}
}

\begin{document}
\raggedright
\begin{sloppypar}
\section*{№1}
  \textbf{Условие задачи} \\~\\
  \begin{equation}
    \Delta u = 0
  \end{equation}
  Краевые условия стационарной задачи:
    \begin{ce}
        u(0, y) = 0 \\
        u(x, 3) = -4 \sin \frac{5\pi x}{2} \\
        u(2, y) = 0 \\
        u(x, 0) = 2 \sin \frac{\pi x}{2} \\
    \end{ce}
    
  \textbf{Решение} \\~\\
  Найдем решения вида $u = X(x)Y(y)$.
  \begin{me}
      \Delta u = u_{xx} + u_{yy} = 0 \\
      X''Y + XY'' = 0 \\
      \frac{X''}{X} = - \frac{Y''}{Y} = c
  \end{me}
  Задача Ш-Л для $X$:
  \begin{ce}
      X'' = cX \\
      X(0) = 0 \\
      X(2) = 0
  \end{ce}
  Решение:
  \begin{me}
      c = - \lambda^2 \\
      \lambda_n = \frac{\pi n}{2} \\
      X_n = \sin{\lambda_n x}
  \end{me}
  Найдем $Y_n$.
  \begin{me}
      Y_n'' = \lambda_n^2 Y_n \\
      Y_n = A_n e^{\lambda_n y} + B_n e^{- \lambda_n y}
  \end{me}
  Мы знаем, что:
  \begin{equation}
    u(x, y) = \sum_{n = 1}^\infty X_n Y_n
  \end{equation}
  Используя два оставшихся условия найдем $A_n$ и $B_n$.
  \begin{equation}
    \begin{gathered}
      u(x, 0) = \sum_{n = 1}^\infty \sin{\frac{\pi n x}{2}} (A_n + B_n) = 2\sin{\frac{\pi x}{2}} \\
      A_n + B_n = 
        \begin{cases}
          2, n = 1 \\
          0, n \neq 1
        \end{cases}
    \end{gathered}
  \end{equation}
  
  \begin{equation}
    \begin{gathered}
      u(x, 3) = \sum_{n = 1}^\infty \sin{\frac{\pi n x}{2}} (A_n e^\frac{3\pi n}{2} + B_n e^{- \frac{3\pi n}{2}}) = -4\sin{\frac{5\pi x}{2}}\\
      A_n e^\frac{3\pi n}{2} + B_n e^{- \frac{3\pi n}{2}} = 
        \begin{cases}
          -4, n = 5 \\
          0, n \neq 5
        \end{cases}
    \end{gathered}
  \end{equation}
  Рассмотрим все значения $n$:
  \begin{itemize}
    \item $\mathbf{n \notin \{1, 5\}}$ $A_n = B_n = 0$
    \item $\mathbf{n = 1}$
      \begin{equation}
        \begin{cases}
          A_1 + B_1 = 2 \\
          A_1 e^\frac{3\pi}{2} + B_1 e^{- \frac{3\pi}{2}} = 0
        \end{cases}
        \begin{cases}
          A_1 = 2 - B_1 \\
          2 e^\frac{3\pi}{2} + B_1(e^{- \frac{3\pi}{2}} - e^\frac{3\pi}{2}) = 0
        \end{cases}
        \begin{cases}
          A_1 = 2 - \frac{2 e^\frac{3\pi}{2}}{e^{- \frac{3\pi}{2}} - e^\frac{3\pi}{2}} \\
          B_1 = \frac{2 e^\frac{3\pi}{2}}{e^{- \frac{3\pi}{2}} - e^\frac{3\pi}{2}}
        \end{cases}
      \end{equation}
    \item $\mathbf{n = 5}$
      \begin{equation}
        \begin{cases}
          A_5 + B_5 = 0 \\
          A_5 e^\frac{15\pi}{2} + B_5 e^{- \frac{15\pi}{2}} = -4
        \end{cases}
        \begin{cases}
          A_5 = \frac{4}{e^{- \frac{15\pi}{2}} - e^\frac{15\pi}{2}} \\
          B_5 = \frac{4}{e^\frac{15\pi}{2} - e^{- \frac{15\pi}{2}}}
        \end{cases}
      \end{equation}
  \end{itemize}
  \textbf{Ответ:}
  \begin{equation}
    u(x, y) = \sin{\left(\frac{\pi x}{2}\right)} \frac{2e^{\frac{\pi}{2}(3 - y)} - 2e^{\frac{\pi}{2}(y - 3)}}{e^\frac{3\pi}{2} - e^{- \frac{3\pi}{2}}} + \sin{\left(\frac{5'pi x}{2}\right)} \frac{4e^{- \frac{5\pi y}{2}} - 4e^\frac{5\pi y}{2}}{e^\frac{15 \pi}{2} - e^{- \frac{15 \pi}{2}}}
  \end{equation}
  
  \section*{№2}
  \textbf{Условие задачи} \\~\\
  \begin{equation}
    u_t = u_{xx} + \sin{4x}\sin{x}
  \end{equation}
  Краевые условия:
  \begin{ce}
    u_x(t, 0) = 3 \\
    u_x(t, \pi) = 3
  \end{ce}
  Начальное условие:
  \begin{equation}
    u(0, x) = \cos{3x} + 3x
  \end{equation}
  
  \textbf{Решение} \\~\\
  Представим функцию в виде суммы:
  \begin{equation}
    u = v + w
  \end{equation}
  Так, что:
  \begin{ce}
    w_x(t, 0) = 3 \\
    w_x(t, \pi) = 3
  \end{ce}
  Понятно, что $w(t, x) = 3x$. \\~\\
  
  ------------------------------
  
  \textit{Примечание:} вид $w$ не всегда очевиден. Например, если:
  \begin{ce}
    w_x(t, 0) = -2t \\
    w(t, \pi) = 3
  \end{ce}
  Тогда решение можно найти следующим образом:
  \begin{me}
    \int w_x(t, 0) dx = \int -2t dx \\
    w(t, x) = -2tx + c(t) \\
    w(t, \pi) = -2\pi t + c(t) = 3 \\
    c(t) = 3 + 2\pi t \\
    w(t, x) = -2tx + 2\pi t + 3
  \end{me}
  ------------------------------
  
  Теперь построим задачу для $v$.
  \begin{me}
    v_t = v_{xx} + \sin{4x} \sin{x} \\
    \begin{cases}
      v_x(t, 0) = 0 \\
      v_x(t, \pi) = 0
    \end{cases} \\
    v(0, x) = \cos{3x}
  \end{me}
  Задача Ш-Л:
  \begin{ce}
    X'' = cX \\
    X'(0) = 0 \\
    X(\pi) = 0
  \end{ce}
  Решение:
  \begin{me}
    c = - \lambda^2 \\
    \lambda_n = n \\
    X_n = \cos{\lambda_n x}
  \end{me}
  Выражаем НУ и неоднородную часть через ряд:
  \begin{me}
    v(0, x) = \cos{3x} = \sum_{n = 0}^\infty \phi_n \cos{nx} \\
    \phi_n = \begin{cases}
      1, n = 3 \\
      0, n \neq 3
    \end{cases}
  \end{me}
  \begin{me}
    f(x) = \sin{4x} \sin{x} = \frac{1}{2}(\cos{3x} - \cos{5x}) = \sum_{n = 0}^\infty f_n \cos{nx} \\
    f_n = \begin{cases}
      \frac{1}{2}, n = 3 \\
      - \frac{1}{2}, n = 5 \\
      0, n \notin \{3, 5\}
    \end{cases}
  \end{me}
  Подставим ряды в исходную задачу.
  \begin{me}
    \sum_{n = 1}^\infty X_n T_n' = \sum_{n = 1}^\infty - \lambda_n^2 X_n T_n + \sum_{n = 1}^\infty f_n X_n \\
    \sum_{n = 1}^\infty X_n T(0) = \sum_{n = 1}^\infty \phi_n X_n
  \end{me}
  Закрепив $n$ и сократив обе части уравнений на $X_n$ получим задачу Коши для $T_n$. Решим её для каждого $n$.
  \begin{itemize}
    \item $\mathbf{n \notin \{3, 5\}}$
      \begin{equation}
        \begin{cases}
          T' = -9T \\
          T(0) = 0
        \end{cases}
        \Rightarrow T = 0
      \end{equation}
    \item $\mathbf{n = 3}$
      \begin{me}
        \begin{cases}
          T' = -9T + \frac{1}{2} \\
          T(0) = 1
        \end{cases}
      \end{me}
      \begin{me}
        \frac{dT}{T} = -9dt \\
        ln T = -9t + c(t) \\
        T = c(t)e^{-9t} \\
      \end{me}
      Подставляем в исходное уравнение и находим $c(t)$.
      \begin{me}
        -9ce^{-9t} + c'e^{-9t} = -9ce^{-9t} + \frac{1}{2} \\
        dc = \frac{dt}{2e^{-9t}} \\
        c(t) = \frac{e^{9t}}{18} + c \\
        T = \frac{1}{18} + ce^{-9t}
      \end{me}
      Находим $c$.
      \begin{me}
        T(0) = \frac{1}{18} + c = 1 \\
        c = \frac{17}{18}
      \end{me}
    \item $\mathbf{n = 5}$
      \begin{me}
        \begin{cases}
          T' = -25T - \frac{1}{2} \\
          T(0) = 0
        \end{cases} \\
        T = c(t)e^{-25t} \\
        c'e^{-25t} = - \frac{1}{2} \\
        c(t) = - \frac{e^{25t}}{50} + c \\
        T = - \frac{1}{50} + ce^{-25t} \\
        T(0) = - \frac{1}{50} + c \Rightarrow c = \frac{1}{50}
      \end{me}
  \end{itemize}
  Не забываем прибавить $w(t, x)$.
  
  \textbf{Ответ:}
  \begin{equation}
    u(t, x) = \cos{3x}\left(\frac{1 +17e^{-9t}}{18}\right) + \cos{5x}\left(\frac{e^{-25t} - 1}{50}\right) + 3x
  \end{equation}
  
  \section*{№3}
  \textbf{Условие задачи} \\~\\
  
  \begin{equation}
    u_{tt} = 9u_{xx}
  \end{equation}
  Начальные условия:
  \begin{ce}
    u(0, x) = 1 - 7\sin{\frac{9\pi x}{2}} \\
    u_t(0, x) = 2\sin{\frac{3\pi x}{2}} - 3x
  \end{ce}
  Греничные условия:
  \begin{ce}
    u(t, 0) = 1 \\
    u_x(t, 1) = -3t
  \end{ce}
  
  \textbf{Решение} \\~\\
  \begin{me}
    u = v + w \\
    \begin{cases}
      w(t, 0) = 1 \\
      w_x(t, 1) = -3t
    \end{cases}
  \end{me}
  
  \begin{me}
    w(t, x) = -3tx + c(t) \\
    w(t, 0) = c = 1 \\
    w(t, x) = -3tx + 1
  \end{me}
  
  \begin{me}
    v_{tt} = 9v_{xx} \\
    \begin{cases}
      v(0, x) = -7\sin{\frac{9\pi x}{2}} \\
      v_t(0, x) = 2\sin{\frac{3 \pi x}{2}}
    \end{cases} 
    \begin{cases}
      v(t, 0) = 0 \\
      v_x(t, 1) = 0
    \end{cases}
  \end{me}
  
  Задача Ш-Л:
  \begin{ce}
    X'' = cX \\
    X(0) = 0 \\
    X_x(1) = 0
  \end{ce}
  \begin{me}
    c = \lambda^2 \\
    \lambda_n = \frac{\pi n}{2} \\
    X_n = \sin{\lambda_n x}
  \end{me}
  
  \begin{equation}
    T_n = A_ne^{3\lambda_n t} + B_ne^{- 3\lambda_n t}
  \end{equation}

  Выражаем НУ через ряд:
  \begin{me}
    v(0, x) = -7\sin{\frac{9\pi x}{2}} = \sum_{n = 1}^\infty \phi_n \sin{\frac{\pi n x}{2}} \\
    \phi_n = \begin{cases}
      -7, n = 9 \\
      0, n \neq 9
    \end{cases}
  \end{me}
  \begin{me}
    v_t(0, x) = 2\sin{\frac{3\pi x}{2}} = \sum_{n = 1}^\infty \psi_n \sin{\frac{\pi n x}{2}} \\
    \psi_n = \begin{cases}
      2, n = 3 \\
      0, n \neq 3
    \end{cases}
  \end{me}
  Находим $A_n$ и $B_n$ для каждого $n$.
  \begin{itemize}
    \item $\mathbf{n \notin \{3, 9\}}$
    \begin{equation}
      \begin{cases}
        (A_n + B_n)\sin{\frac{\pi n x}{2}} = 0 \\
        (\frac{3\pi n}{2}A_n - \frac{3\pi n}{2}B_n)\sin{\frac{\pi n x}{2}} = 0
      \end{cases}
      \Rightarrow A_n = B_n = 0  
    \end{equation}
    \item $\mathbf{n = 3}$
    \begin{equation}
      \begin{cases}
        (A_3 + B_3)\sin{\frac{3\pi x}{2}} = 0 \\
        \frac{9\pi}{2}(A_3 - B_3)\sin{\frac{3\pi x}{2}} = 2\sin{\frac{3\pi x}{2}}     
      \end{cases}
      \begin{cases}
        A_3 + B_3 = 0 \\
        A_3 - B_3 = \frac{2}{9\pi}
      \end{cases}
      \begin{cases}
        A_3 = \frac{1}{9\pi} \\
        B_3 = - \frac{1}{9\pi}
      \end{cases}
    \end{equation}
    \item $\mathbf{n = 9}$
    \begin{equation}
      \begin{cases}
        (A_9 + B_9)\sin{\frac{9\pi x}{2}} = -7\sin{\frac{9\pi x}{2}} \\
        \frac{27\pi}{2}(A_9 - B_9)\sin{\frac{9\pi x}{2}} = 0
      \end{cases}
      \begin{cases}
        A_9 + B_9 = -7 \\
        A_9 - B_9 = 0
      \end{cases}
      A_9 = B_9 = - \frac{7}{2}
    \end{equation}
  \end{itemize}
  
  \textbf{Ответ:}
  \begin{equation}
    u(t, x) = \frac{1}{9\pi} \sin{\left(\frac{3 \pi x}{2}\right)} (e^\frac{9\pi t}{2} - e^{- \frac{9\pi t}{2}}) - \frac{7}{2} \sin{\left(\frac{9 \pi x}{2}\right)}(e^\frac{27\pi t}{2} + e^{- \frac{27\pi t}{2}}) - 3tx + 1
  \end{equation}      
\end{sloppypar}
\end{document}