\documentclass[11pt]{article}
\usepackage[utf8]{inputenc}
\usepackage[russian]{babel}
\usepackage{amsmath}
\usepackage{ntheorem}
\usepackage{enumitem}
\usepackage{tabulary}
\usepackage[left=2cm, right=2cm, top=2cm]{geometry}
\usepackage[T1]{fontenc}

\newtheorem*{df}{Определение}
\newtheorem*{theorem}{Теорема}
\theoremstyle{break}
\newtheorem*{example}{Пример:}
\newtheorem*{proof}{Доказательство:}

\begin{document}
\raggedright
\begin{sloppypar}
  \section*{Введение}
  Общий вид задачи оптимизации:
  \begin{enumerate}
    \item \textbf{Задача оптимального выбора}
    
      $\min (\max) f(x), x \in X$ - множество операторов.

      Если $X = R^n$ - задача безусловной минимизации (задача без ограничений)
    \item \textbf{Общая задача математического программирования}

      $\min f_0(x)$ - функция m переменных. $x \in X_0 \subset R^n$
      
      $f_i(x) \leq 0, i \in \overline{1, m_1}$
      
      $f_i(x) = 0, i \in \overline{m_1 + 1, m}$
    \item \textbf{Задача вариационного исчисления}
    
      $\min J(x) = \int_{t_0}^T L(t, x(t), x'(t))$
      
      $x(t_0) = x_0, x(T) = x_1$
  \end{enumerate}
  \subsection*{Задача планирования производства}
  Переработка $m$ видов ингредиентов (ресурсов)

  $b_i$ – объем $i$-го ресурса
  
  $n$ технологий
  
  $a_{i, j}$ – затраты $i$–го ресурса при использовании $j$–ой технологии с единичной интенсивностью (например, за единицу времени)
  
  $c_j$ – ценность за ед. времени $j$-го способа

  Требуется спланировать производство так, чтобы не выходя за рамки отпущенных ресурсов получить конечную продукцию максимальной суммарной ценности.
  
  Ищем интенсивность $j$-ого способа производства $x_j$.
  
  Ищем $x = (x1 \dots xn)$ - план производства, который максимизирует суммарную ценность.
  
  $\begin{cases}
    \max \sum_{j = 1}^n c_j x_j \\
    \sum_{j = 1}^n a_{i, j} x_j \leq b_i, i \in \overline{1, m} \\
    x_j \geq 0, j \in \overline{1, n}
  \end{cases}$
  
  \subsection*{Задача диеты}
  $m$ полезных веществ
  
  $b_i$ - минимальное количество $i$-го вещества
  
  $n$ продуктов питания

  $a_{i, j}$ - количество $i$–го вещества в единице веса $j$-го продукта

  $c_j$ - цена единицы $j$-го продукта

  Требуется найти количество продуктов $x_j$
  
  $\begin{cases}
    \min \sum_{j = 1}^n c_j x_j \\
    \sum_{j = 1}^n a_{i, j} x_j \geq b_i, i \in \overline{1, m} \\
    x_j \geq 0, j \in \overline{1, n}
  \end{cases}$
  
  \subsection*{Транспортная задача}
  $m$ пунктов производства
  
  $a_i$ - количество продукта в $i$-м пункте 

  $n$ потребителей

  $b_j$ - потребность $j$-го потребителя
  
  $c_{i, j}$ - стоимость перевозки из пункта $i$ в пункт $j$ единицы продукта
  
  Требуется организовать перевозки так, чтобы:
  \begin{enumerate}
    \item из каждого пункта производства вывезти весь имеющийся там продукт 
    \item полностью насытить потребности каждого потребителя
    \item суммарные транспортные затраты были минимальны
  \end{enumerate}
  
  Определить объемы перевозок $x_{i, j}$
  
  $\begin{cases}
    \min \sum_{i = 1}^m \sum_{j = 1}^n c_{i, j} x_{i, j} \\
    \sum_{j = 1}^n x_{i, j} = a_i, i \in \overline{1, m} \\
    \sum_{i = 1}^m x_{i, j} = b_j, j \in \overline{1, n} \\
    x_{i, j} \geq 0, \forall i, j
  \end{cases}$
  
  Все три задачи оптимизационные, во всех надо найти оптимум линейной функции. Существуют ограничения в виде неравенств и равенств. В ограничениях левая часть – линейная функция. Есть условия неотрицательных переменных ($x_{i, j} > 0$)
  
  Примеры задач линейного программирования вкладываются в общую схему задач математического программирования.
  
  \section*{Общая задача линейного программирования}
  $\begin{cases}
    \min(\max) \sum_{j = 1}^n c_j x_j \\
    \sum_{j = 1}^n a_{i, j} x_j \{=, \leq, \geq\} b_i, i \in \overline{1, m} \\
    x_j \geq 0, j \in \overline{1, m}
  \end{cases}$
  
  \textbf{Целевая функция} - функция, которая минимизируется или максимизируется.
  
  \textbf{Вектор цели} $c = (c_1 \dots с_n)$ определяет целевую функцию.
  
  \textbf{Ограничения} могут быть равенствами или неравенствами.
  
  
  \textbf{Матрица задачи (условий)} $A = (a_{i, j})_{m \times n}$.
  	
  \textbf{Вектор правых частей (ограничений)} $B = (b_1 \dots b_m)$.
  
  \textbf{Условие неотрицательности переменных} $x_j >= 0$.

  \textbf{План задачи} $x = (x_1 \dots x_n)$ - допустимый, если удовлетворяет всем ограничениям.
  
  \textbf{Допустимое множество} $X$ – множество всех допустимых планов задачи.

  $\overline{x} = (\overline{x_1} \dots \overline{x_n})$ - \textbf{Оптимальный план} , если $\forall x \in X: \sum_{j = 1}^n c_j \overline{x_j} \leq \sum_{j = 1}^n c_j x_j$

$\sum_{j = 1}^n c_j \overline{x_j}$ - \textbf{Оптимальное значение задачи}

\textbf{Решение задачи линейного программирования} - найти хотя бы один оптимальный план и вычислить оптимальное значение.

\section*{Частные формы задачи ЛП}
\begin{enumerate}
  \item Планирования производства
  
    $\begin{cases}
      \max \sum_{j = 1}^n c_j x_j \\
      \sum_{j = 1}^n a_{i, j} x_j = b_i, \forall i \\
      x_j \geq 0, \forall j
    \end{cases}$
  \item Каноническая задача
  
    $\begin{cases}
      \min \sum_{j = 1}^n c_j x_j \\
      \sum_{j = 1}^n a_{i, j} x_j = b_i, \forall i \\
      x_j \geq 0, \forall j
    \end{cases}$
  \item Основная задача
    
    $\begin{cases}
      \min \sum_{j = 1}^n c_j x_j \\
      \sum_{j = 1}^n a_{i, j} x_j \leq b_i, \forall i \\
    \end{cases}$
\end{enumerate}

Существуют правила перехода от одной задачи к другой (формы эквивалентны).
\begin{enumerate}
  \item важна с точки зрения приложений
  \item решается алгебраическими методами, приводим задачи к этому виду для решения
  \item важна при рассмотрении теоретических вопросов
\end{enumerate}

\begin{itemize}
\item Матричная запись:

 $X, C, B$ - вектор-столбцы

$\begin{cases}
  \max C^T X \\
  AX \leq B \\
  X \geq 0
\end{cases}$

\item Векторная запись:

$A = \begin{bmatrix}
  a_1 \\ \vdots \\ a_m
\end{bmatrix}$ - набор векторов (строк) 

$\begin{cases}
  \max (C, X) \\
  (a_i, X) \leq b_i, \forall i \\
  X \geq 0
\end{cases}$

\item Запись через столбцы:

$A = (A_1 \dots A_n)$ - набор столбцов

$\begin{cases}
  \max (C, X) \\
  \sum_{j = 1}^n A_j x_j \leq B \\
  X \geq 0
\end{cases}$
\end{itemize}

\subsection*{Правила перехода}
\begin{enumerate}
  \item $\max\limits_{x \in S} f(x) = - \min\limits_{x \in S} (- f(x))$
  
    $x^*$ - точка максимума $f(x)$
  
    $\forall x \in S: f(x^*) \geq f(x)$
  
    $\forall x \in S: - f(x^*) \leq - f(x)$
  
    $x^*$ - точка минимума $- f(x)$
  
  \item $f_i(x) \geq 0 \sim - f_i(x) \leq 0$
  
  \item $(a_i, x) \leq b_i$
  
    Добавим переменную $x_{n + i} = b_i - (a_i, x) \geq 0$
    
    $(a_i, x) \leq b_i \sim
    \begin{cases}
      (a_i, x) + x_{n + i} = b_i \\
      x_{n + i} \geq 0    
    \end{cases}$
    
  \item $(a_i, x) = b_i \sim
    \begin{cases}
      (a_i, x) \leq b_i \\
      - (a_i, x) \leq b_i
    \end{cases}$
    
  \item $a_{i, 1}x_1 + \dots + a_{i, n}x_n = b_i$
  
    Пусть $a_{i, 1} \neq 0$
    
    $x_1 = \frac{1}{a_{i, 1}}(b_i - a_{i, 2}x_2 - \dots - a_{i, n}x_n)$
    
    $x_i \geq 0 \Rightarrow a_{i, 2}x_2 + \dots + a_{i, n}x_n \leq b_i$
    
    Применим метод Жордана-Гаусса. Количество ограничений сократится.
    
  \item $x_j \geq 0 \sim - x_j \leq 0 \sim (a_j, x) \leq b_j, b_j = 0, a_j = -1$
  
  \item $x_j$ - свободная переменная.
  
    Замена $x_j = x'_j - x''_j, x'_j, x''_j \geq 0$
\end{enumerate}

\end{sloppypar}
\end{document}
