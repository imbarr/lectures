\documentclass[11pt]{article}
\usepackage[utf8]{inputenc}
\usepackage[russian]{babel}
\usepackage{amsmath}
\usepackage{ntheorem}
\usepackage{enumitem}
\usepackage{changepage}
\usepackage{tabulary}
\usepackage[left=2cm, right=2cm, top=2cm]{geometry}
\usepackage[T1]{fontenc}

\newenvironment{inlinedf}[1]{
  \underline{\textbf{Опр.}}\ \ \textbf{#1} - }{
}

\newenvironment{df}[1]{
  \underline{\textbf{Опр.}}\ \ \textbf{#1}
  
  \begin{adjustwidth}{1em}{}
}{
  \end{adjustwidth}
}

\newenvironment{proof}{
  \textit{Доказательство:}
    
  \begin{adjustwidth}{1em}{}
}{
  \end{adjustwidth}
}

\newenvironment{statement}{
  \underline{\textbf{Утв.}}\ \ }{
  
}

\newenvironment{theorem}{
  \underline{\textbf{Теорема}}\ \ }{
  
}

\newenvironment{mlstatement}{
  \underline{\textbf{Утв.}}\ \ 
  
  \begin{adjustwidth}{1em}{}
}{
  \end{adjustwidth}
}

\newenvironment{example}{
  \textit{Пример:}
    
  \begin{adjustwidth}{1em}{}
}{
  \end{adjustwidth}
}

\begin{document}
\raggedright
\begin{sloppypar}
  \section*{Введение}
  Общий вид задачи оптимизации:
  \begin{enumerate}
    \item \textbf{Задача оптимального выбора}
    
      $\min (\max) f(x), x \in X$ - множество операторов.

      Если $X = R^n$ - задача безусловной минимизации (задача без ограничений)
    \item \textbf{Общая задача математического программирования}

      $\min f_0(x)$ - функция m переменных. $x \in X_0 \subset R^n$
      
      $f_i(x) \leq 0, i \in \overline{1, m_1}$
      
      $f_i(x) = 0, i \in \overline{m_1 + 1, m}$
    \item \textbf{Задача вариационного исчисления}
    
      $\min J(x) = \int_{t_0}^T L(t, x(t), x'(t))$
      
      $x(t_0) = x_0, x(T) = x_1$
  \end{enumerate}
  \subsection*{Задача планирования производства}
  Переработка $m$ видов ингредиентов (ресурсов)

  $b_i$ – объем $i$-го ресурса
  
  $n$ технологий
  
  $a_{i, j}$ – затраты $i$–го ресурса при использовании $j$–ой технологии с единичной интенсивностью (например, за единицу времени)
  
  $c_j$ – ценность за ед. времени $j$-го способа

  Требуется спланировать производство так, чтобы не выходя за рамки отпущенных ресурсов получить конечную продукцию максимальной суммарной ценности.
  
  Ищем интенсивность $j$-ого способа производства $x_j$.
  
  Ищем $x = (x1 \dots xn)$ - план производства, который максимизирует суммарную ценность.
  
  $\begin{cases}
    \max \sum_{j = 1}^n c_j x_j \\
    \sum_{j = 1}^n a_{i, j} x_j \leq b_i, i \in \overline{1, m} \\
    x_j \geq 0, j \in \overline{1, n}
  \end{cases}$
  
  \subsection*{Задача диеты}
  $m$ полезных веществ
  
  $b_i$ - минимальное количество $i$-го вещества
  
  $n$ продуктов питания

  $a_{i, j}$ - количество $i$–го вещества в единице веса $j$-го продукта

  $c_j$ - цена единицы $j$-го продукта

  Требуется найти количество продуктов $x_j$
  
  $\begin{cases}
    \min \sum_{j = 1}^n c_j x_j \\
    \sum_{j = 1}^n a_{i, j} x_j \geq b_i, i \in \overline{1, m} \\
    x_j \geq 0, j \in \overline{1, n}
  \end{cases}$
  
  \subsection*{Транспортная задача}
  $m$ пунктов производства
  
  $a_i$ - количество продукта в $i$-м пункте 

  $n$ потребителей

  $b_j$ - потребность $j$-го потребителя
  
  $c_{i, j}$ - стоимость перевозки из пункта $i$ в пункт $j$ единицы продукта
  
  Требуется организовать перевозки так, чтобы:
  \begin{enumerate}
    \item из каждого пункта производства вывезти весь имеющийся там продукт 
    \item полностью насытить потребности каждого потребителя
    \item суммарные транспортные затраты были минимальны
  \end{enumerate}
  
  Определить объемы перевозок $x_{i, j}$
  
  $\begin{cases}
    \min \sum_{i = 1}^m \sum_{j = 1}^n c_{i, j} x_{i, j} \\
    \sum_{j = 1}^n x_{i, j} = a_i, i \in \overline{1, m} \\
    \sum_{i = 1}^m x_{i, j} = b_j, j \in \overline{1, n} \\
    x_{i, j} \geq 0, \forall i, j
  \end{cases}$
  
  Все три задачи оптимизационные, во всех надо найти оптимум линейной функции. Существуют ограничения в виде неравенств и равенств. В ограничениях левая часть – линейная функция. Есть условия неотрицательных переменных ($x_{i, j} > 0$)
  
  Примеры задач линейного программирования вкладываются в общую схему задач математического программирования.
  
  \section*{Общая задача линейного программирования}
  $\begin{cases}
    \min(\max) \sum_{j = 1}^n c_j x_j \\
    \sum_{j = 1}^n a_{i, j} x_j \{=, \leq, \geq\} b_i, i \in \overline{1, m} \\
    x_j \geq 0, j \in \overline{1, m}
  \end{cases}$
  
  \textbf{Целевая функция} - функция, которая минимизируется или максимизируется.
  
  \textbf{Вектор цели} $c = (c_1 \dots с_n)$ определяет целевую функцию.
  
  \textbf{Ограничения} могут быть равенствами или неравенствами.
  
  
  \textbf{Матрица задачи (условий)} $A = (a_{i, j})_{m \times n}$.
  	
  \textbf{Вектор правых частей (ограничений)} $B = (b_1 \dots b_m)$.
  
  \textbf{Условие неотрицательности переменных} $x_j >= 0$.

  \textbf{План задачи} $x = (x_1 \dots x_n)$ - допустимый, если удовлетворяет всем ограничениям.
  
  \textbf{Допустимое множество} $X$ – множество всех допустимых планов задачи.

  $\overline{x} = (\overline{x_1} \dots \overline{x_n})$ - \textbf{Оптимальный план} , если $\forall x \in X: \sum_{j = 1}^n c_j \overline{x_j} \leq \sum_{j = 1}^n c_j x_j$

$\sum_{j = 1}^n c_j \overline{x_j}$ - \textbf{Оптимальное значение задачи}

\textbf{Решение задачи линейного программирования} - найти хотя бы один оптимальный план и вычислить оптимальное значение.

\section*{Частные формы задачи ЛП}
\begin{enumerate}
  \item Планирования производства
  
    $\begin{cases}
      \max \sum_{j = 1}^n c_j x_j \\
      \sum_{j = 1}^n a_{i, j} x_j = b_i, \forall i \\
      x_j \geq 0, \forall j
    \end{cases}$
  \item Каноническая задача
  
    $\begin{cases}
      \min \sum_{j = 1}^n c_j x_j \\
      \sum_{j = 1}^n a_{i, j} x_j = b_i, \forall i \\
      x_j \geq 0, \forall j
    \end{cases}$
  \item Основная задача
    
    $\begin{cases}
      \min \sum_{j = 1}^n c_j x_j \\
      \sum_{j = 1}^n a_{i, j} x_j \leq b_i, \forall i \\
    \end{cases}$
\end{enumerate}

Существуют правила перехода от одной задачи к другой (формы эквивалентны).
\begin{enumerate}
  \item важна с точки зрения приложений
  \item решается алгебраическими методами, приводим задачи к этому виду для решения
  \item важна при рассмотрении теоретических вопросов
\end{enumerate}

\begin{itemize}
\item Матричная запись:

 $X, C, B$ - вектор-столбцы

$\begin{cases}
  \max C^T X \\
  AX \leq B \\
  X \geq 0
\end{cases}$

\item Векторная запись:

$A = \begin{bmatrix}
  a_1 \\ \vdots \\ a_m
\end{bmatrix}$ - набор векторов (строк) 

$\begin{cases}
  \max (C, X) \\
  (a_i, X) \leq b_i, \forall i \\
  X \geq 0
\end{cases}$

\item Запись через столбцы:

$A = (A_1 \dots A_n)$ - набор столбцов

$\begin{cases}
  \max (C, X) \\
  \sum_{j = 1}^n A_j x_j \leq B \\
  X \geq 0
\end{cases}$
\end{itemize}

\subsection*{Правила перехода}
\begin{enumerate}
  \item $\max\limits_{x \in S} f(x) = - \min\limits_{x \in S} (- f(x))$
  
    $x^*$ - точка максимума $f(x)$
  
    $\forall x \in S: f(x^*) \geq f(x)$
  
    $\forall x \in S: - f(x^*) \leq - f(x)$
  
    $x^*$ - точка минимума $- f(x)$
  
  \item $f_i(x) \geq 0 \sim - f_i(x) \leq 0$
  
  \item $(a_i, x) \leq b_i$
  
    Добавим переменную $x_{n + i} = b_i - (a_i, x) \geq 0$
    
    $(a_i, x) \leq b_i \sim
    \begin{cases}
      (a_i, x) + x_{n + i} = b_i \\
      x_{n + i} \geq 0    
    \end{cases}$
    
  \item $(a_i, x) = b_i \sim
    \begin{cases}
      (a_i, x) \leq b_i \\
      - (a_i, x) \leq b_i
    \end{cases}$
    
  \item $a_{i, 1}x_1 + \dots + a_{i, n}x_n = b_i$
  
    Пусть $a_{i, 1} \neq 0$
    
    $x_1 = \frac{1}{a_{i, 1}}(b_i - a_{i, 2}x_2 - \dots - a_{i, n}x_n)$
    
    $x_i \geq 0 \Rightarrow a_{i, 2}x_2 + \dots + a_{i, n}x_n \leq b_i$
    
    Применим метод Жордана-Гаусса. Количество ограничений сократится.
    
  \item $x_j \geq 0 \sim - x_j \leq 0 \sim (a_j, x) \leq b_j, b_j = 0, a_j = -1$
  
  \item $x_j$ - свободная переменная.
  
    Замена $x_j = x'_j - x''_j, x'_j, x''_j \geq 0$
\end{enumerate}

\section*{Геометрическая задача ЛП на плоскости и графический метод решения}
$z = c_1 x_1 + c_2 x_2 \rightarrow \max$

$\begin{cases}
  a_{1, 1}x_1 + a_{1, 2}x_2 \leq b_1 \\
  \vdots \\
  a_{m, 1}x_1 + a_{m, 2}x_2 \leq b_m
\end{cases}$

$x_i \geq 0 \forall i$

\begin{enumerate}
  \item Построение допустимого множества $X$.
  
  $a_{i, 1}x_1 + a_{i, 2}x_2 \leq b_i$
  
  Множество точек, удовлетворяющих неравенству - полуплоскость. Построим прямую $a_{i, 1}x_1 + a_{i, 2}x_2 = b_i$. Вектор нормали $(a_{i, 1}, a_{i, 2})$ направлен в искомую полуплоскость.
  
  Также учитываем условия неотрицательности переменных.
  
  Множество $X$ - пересечение всех полученных полуплоскостей. Возможны три случая:
  \begin{enumerate}
    \item $X$ - многоугольник
    \item $X$ - неограниченное многоугольное множество
    \item $X$ - пустое множество (решений нет)
  \end{enumerate}

  \item Поиск $\max z = \max (c, x)$ с помощью линий уровня.
  
  $(c, x) = z_0$ - начальная прямая. Как правило, $z_0 = 0$.
  
  $c$ - её вектор нормали.
  
  Возьмем $z_1 > z_0$. Прямая $(c, x) = z_1$ параллельно сдвинута в сторону вектора нормали (в задаче на $\min$ сдвигаем в обратную сторону).
  
  Сдвигаем прямую до крайнего положения $(c, x) = f^*$. Множество оптимальных планов $x^*$ - пересечение $X$ и прямой $(c, x) = f^*$.
  
  Если $X$ - неограниченное множество, то возможна (зависит от направления вектора $c$) ситуация, когда $f^* = \infty$.
\end{enumerate}

\section*{Геометрические свойства задачи ЛП в пространстве $R^n$}
\begin{df}{Выпуклая комбинация}

  $x - x_1$ - часть вектора $x_2 - x_1$.
  
  $x - x_1 = \alpha(x_2 - x_1)$.
  
  Значение $\alpha$ задает положение точки $x$ на отрезке $[x_1, x_2]$.
  
  $x = (1 - \alpha)x_1 + \alpha x_2$ - уравнение отрезка $[x_1, x_2]$ при $0 \leq \alpha \leq 1$ (выпуклая комбинация точек $x_1$, $x_2$).
\end{df}

\begin{df}{Выпуклое множество}
  $S$ - выпуклое, если любой отрезок с концами в $S$ целиком содержится в $S$.
  
  $\forall x_1, x_2 \in S: [x_1, x_2] \subset S$
  
  Или:
  
  $\forall x_1, x_2 \in S, \alpha \in [0, 1]: x = (1 - \alpha)x_1 + \alpha x_2 \in S$
\end{df}

\begin{statement}
  Непустое пересечение выпуклых множеств - выпуклое.
\end{statement}
\begin{proof}
  $S = S_1 \cap S_2$

  Возьмем $x_1, x_2 \in S$
  
  $x_1, x_2 \in S_i \Rightarrow [x_1, x_2] \subset S_i \Rightarrow [x_1, x_2] \subset S_1 \cap S_2$
\end{proof}

\begin{df}{Гиперплоскость}
  $x \in R^m$

  $H = \{x|(a, x) = b\}$ - гиперплоскость
  
  $a$ - нормаль гиперплоскости
\end{df}

\begin{statement}
  Гиперплоскость - выпуклое множество.
\end{statement}
\begin{proof}
  $x_1, x_2 \in H, \alpha \in [0, 1]$
  
  $x = (1 - \alpha)x_1 + \alpha x_2$
  
  Скалярное произведение линейно $\Rightarrow$
  
  $(a, x) = (1 - \alpha)(a, x_1) + \alpha (a, x_2) = (1 - \alpha) b + \alpha b = b$
  
  $\Rightarrow x \in H$
\end{proof}

$H$ делит пространство на две полуплоскости:

$H^+ = \{x|(a, x) \geq b\}$

$H^- = \{x|(a, x) < b\}$

\begin{statement}
  $H^+$ и $H^-$ - выпуклые множества.
\end{statement}
\begin{proof}
  $x_1, x_2 \in H^+, \alpha \in [0, 1]$
  
  $x = (1 - \alpha)x_1 + \alpha x_2$
  
  $(a, x) = (1 - \alpha)(a, x_1) + \alpha (a, x_2) = (1 - \alpha)(b + \theta_1) + \alpha (b + \theta_2) = b + (1 - \alpha)\theta_1  + \alpha\theta_2  \geq b$
  
  где $\theta_1, \theta_2 \geq 0$
  
  Для $H^-$ аналогично.
\end{proof}

$H^-_i = \{x|(a_i, x) \leq b_i\}$
  
$X = \bigcap\limits_{i = 1}^m H^-_i$ - допустимое множество основной задачи (многогранник решений)

\begin{statement}
  $X$ - выпуклое множество.
\end{statement}

\begin{df}{Предельная точка}
  $\overline{x} \in X$ - предельная точка, если:
  
  $\exists \{x_k\} \subset X: \forall k \in N: x_k \neq \overline{x}$ и $x_k \xrightarrow[k \to \infty]{} \overline{x}$
\end{df}

Скалярное произведение непрерывно $\Rightarrow (a_i, x_k) \xrightarrow[k \to \infty]{} (a_i, \overline{x}) \leq b_i$

Значит $X$ - замкнутое множество.

\begin{df}{k-грань}
  $x$ - k-грань, если:

  $\exists \{i_1 \dots i_k\} \subset \overline{1, n}$, такое, что:
  
  \begin{enumerate}
    \item $a_{i_l}$ - линейно независимы
    \item $\forall l \in \overline{1, k}: (a_{i_l}, x) = b_{i_l}$
    \item $\forall i \in \overline{1, n}: (a_i, x) \leq b_i$
  \end{enumerate}
\end{df}

\begin{inlinedf}{Вершина}
  $n$-грань
\end{inlinedf}

\begin{mlstatement}
  Пусть $X \neq \O$, $r = r(A)$. Тогда $\forall k \in \overline{1, r}: \exists k$-грань
  
  \textit{(Без доказательства)}
\end{mlstatement}

%TODO непонятное утверждение

\begin{df}{Выпуклая комбинация}
  Выпуклая комбинация точек $x_1 \dots x_k$ - $\sum_{i = 1}^k \alpha_i x_i$, где:
  
  $\forall i: \alpha_i = 0$
  
  $\sum_{i = 1}^k \alpha_i = 1$
\end{df}

\begin{statement}
  Выпуклое множество содержит все выпуклые комбинации своих точек.
\end{statement}
\begin{proof}
  Пусть $S$ - выпуклое множество, $x_1 \dots x_m \in S$
  
  $x = \sum_{i = 1}^m \alpha_i x_i$
  
  Докажем, что $x \in S$ индукцией по числу точек $m$:
  
  \textit{Б.И.} $k = 2$
  
  Множество значений $x$ - отрезок. Из определения выпуклого множества $x \in S$.
  
  \textit{Ш.И.}
  
  Рассмотрим три случая:
  \begin{enumerate}
    \item $\alpha_m = 0$
    
    Количество слагаемых - $m - 1$. Выполнено П.И.
    \item $\alpha_m = 1$
    
    $\sum_{i = 1}^m \alpha_i = 1 \Rightarrow$ одно слагаемое, очевидно.
    \item $0 < \alpha_m < 1$
    
    $x = (1 - \alpha_m) \sum_{i = 1}^{m - 1} \frac{\alpha_i}{1 - \alpha_m} x_i + \alpha_m x_m = (1 - \alpha_m)y + \alpha_m x_m$
    
    $\sum_{i = 1}^{m - 1} \alpha_i = 1 - \alpha_m \Rightarrow \sum_{i = 1}^{m - 1} \frac{\alpha_i}{1 - \alpha_m} = 1 \Rightarrow y$ - выпуклая комбинация $m - 1$ точек и по П.И. $y \in S$
    
    $\Rightarrow x$ - выпуклая комбинация 2-х точек из $S \Rightarrow x \in S$
  \end{enumerate}
\end{proof}

\begin{df}{Выпуклая оболочка}
  Выпуклая оболочка множества $M$ - множество всех выпуклых комбинаций точек из $M$.
\end{df}

Свойства $coM$:
\begin{enumerate}
  \item $coM$ - выпуклое множество.
  
    $x_1, x_2 \in coM$
    
    $x_1 = \sum_{i = 1}^k \alpha_i x_i$
    
    $x_2 = \sum_{j = 1}^m \beta_j x_j^*$
    
    $x = \lambda x_1 + (1 - \lambda)x_2$
    
    Легко проверить, что $x$ - выпуклая комбинация $x_1 \dots x_k, x_1^* \dots x_m^*$.
  \item $M \subset coM$
  
  Любая точка множества - выпуклая комбинация из одного слагаемого.
  \item $\forall S$ - выпуклое множество: $M \subset S \Rightarrow coM \subset S$
  
    $S$ - выпуклое, значит, оно содержит все выпуклые комбинации своих точек.
    
    $M \subset S \Rightarrow S$ содержит все выпуклые комбинации точек из $M \Rightarrow coM \subset S$
\end{enumerate}

\begin{example}
  Треугольник - выпуклая оболочка его трех вершин.
  
  \begin{proof}
    $x_1, x_2, x_3$ - вершины.
    
    $y = \alpha x_1 + (1 - \alpha)x_2$ - отрезок $[x_1, x_2]$
    
    $x = \beta y + (1 - \beta)x_3$ - треугольник
    
    $x = \beta \alpha x_1 + \beta(1 - \alpha)x_2 + (1 - \beta)x_3$
    
    Сумма коэффициентов равна единице $\Rightarrow x$ - выпуклая комбинация.
  \end{proof}
  
  Многоугольник - выпуклая оболочка своих вершин.
\end{example}

\begin{df}{Крайняя точка}
  $\overline{x} \in M$ - крайняя (угловая) точка, если ее нельзя представить в виде нетривиальной ($\alpha > 0$) выпуклой комбинации двух различных точек из $M$.
\end{df}

\begin{example}
  Крайние точки многоугольника - его вершины. У сферы все точки являются крайними.
\end{example}

\begin{theorem}
  Любое непустое выпуклое замкнутое ограниченное множество из $R^n$ можно представить как выпуклую оболочку своих крайних точек.
\end{theorem}
%TODO доказательство теоремы



\end{sloppypar}
\end{document}
