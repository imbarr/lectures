\documentclass[11pt]{article}
\usepackage[utf8]{inputenc}
\usepackage[russian]{babel}
\usepackage{amsmath}
\usepackage{ntheorem}
\usepackage{enumitem}
\usepackage{changepage}
\usepackage{tabulary}
\usepackage[left=2cm, right=2cm, top=2cm]{geometry}
\usepackage[T1]{fontenc}

\newenvironment{inlinedf}[1]{
  \underline{\textbf{Опр.}}\ \ \textbf{#1} - }{
}

\newenvironment{df}[1]{
  \underline{\textbf{Опр.}}\ \ \textbf{#1}
  
  \begin{adjustwidth}{1em}{}
}{
  \end{adjustwidth}
}

\newenvironment{proof}{
  \textit{Доказательство:}
    
  \begin{adjustwidth}{1em}{}
}{
  \end{adjustwidth}
}

\newenvironment{statement}{
  \underline{\textbf{Утв.}}\ \ }{
  
}

\newenvironment{theorem}{
  \underline{\textbf{Теорема}}\ \ }{
  
}

\newenvironment{mlstatement}[1]{
  \underline{\textbf{Утв.}}\ \ \textbf{#1} 
  
  \begin{adjustwidth}{1em}{}
}{
  \end{adjustwidth}
}

\newenvironment{example}{
  \textit{Пример:}
    
  \begin{adjustwidth}{1em}{}
}{
  \end{adjustwidth}
}

\begin{document}
\raggedright
\begin{sloppypar}
  \section*{Введение}
  Общий вид задачи оптимизации:
  \begin{enumerate}
    \item \textbf{Задача оптимального выбора}
    
      $\min (\max) f(x), x \in X$ - множество операторов.

      Если $X = R^n$ - задача безусловной минимизации (задача без ограничений)
    \item \textbf{Общая задача математического программирования}

      $\min f_0(x)$ - функция m переменных. $x \in X_0 \subset R^n$
      
      $f_i(x) \leq 0, i \in \overline{1, m_1}$
      
      $f_i(x) = 0, i \in \overline{m_1 + 1, m}$
    \item \textbf{Задача вариационного исчисления}
    
      $\min J(x) = \int_{t_0}^T L(t, x(t), x'(t))$
      
      $x(t_0) = x_0, x(T) = x_1$
  \end{enumerate}
  \subsection*{Задача планирования производства}
  Переработка $m$ видов ингредиентов (ресурсов)

  $b_i$ – объем $i$-го ресурса
  
  $n$ технологий
  
  $a_{i, j}$ – затраты $i$–го ресурса при использовании $j$–ой технологии с единичной интенсивностью (например, за единицу времени)
  
  $c_j$ – ценность за ед. времени $j$-го способа

  Требуется спланировать производство так, чтобы не выходя за рамки отпущенных ресурсов получить конечную продукцию максимальной суммарной ценности.
  
  Ищем интенсивность $j$-ого способа производства $x_j$.
  
  Ищем $x = (x1 \dots xn)$ - план производства, который максимизирует суммарную ценность.
  
  $\begin{cases}
    \max \sum_{j = 1}^n c_j x_j \\
    \sum_{j = 1}^n a_{i, j} x_j \leq b_i, i \in \overline{1, m} \\
    x_j \geq 0, j \in \overline{1, n}
  \end{cases}$
  
  \subsection*{Задача диеты}
  $m$ полезных веществ
  
  $b_i$ - минимальное количество $i$-го вещества
  
  $n$ продуктов питания

  $a_{i, j}$ - количество $i$–го вещества в единице веса $j$-го продукта

  $c_j$ - цена единицы $j$-го продукта

  Требуется найти количество продуктов $x_j$
  
  $\begin{cases}
    \min \sum_{j = 1}^n c_j x_j \\
    \sum_{j = 1}^n a_{i, j} x_j \geq b_i, i \in \overline{1, m} \\
    x_j \geq 0, j \in \overline{1, n}
  \end{cases}$
  
  \subsection*{Транспортная задача}
  $m$ пунктов производства
  
  $a_i$ - количество продукта в $i$-м пункте 

  $n$ потребителей

  $b_j$ - потребность $j$-го потребителя
  
  $c_{i, j}$ - стоимость перевозки из пункта $i$ в пункт $j$ единицы продукта
  
  Требуется организовать перевозки так, чтобы:
  \begin{enumerate}
    \item из каждого пункта производства вывезти весь имеющийся там продукт 
    \item полностью насытить потребности каждого потребителя
    \item суммарные транспортные затраты были минимальны
  \end{enumerate}
  
  Определить объемы перевозок $x_{i, j}$
  
  $\begin{cases}
    \min \sum_{i = 1}^m \sum_{j = 1}^n c_{i, j} x_{i, j} \\
    \sum_{j = 1}^n x_{i, j} = a_i, i \in \overline{1, m} \\
    \sum_{i = 1}^m x_{i, j} = b_j, j \in \overline{1, n} \\
    x_{i, j} \geq 0, \forall i, j
  \end{cases}$
  
  Все три задачи оптимизационные, во всех надо найти оптимум линейной функции. Существуют ограничения в виде неравенств и равенств. В ограничениях левая часть – линейная функция. Есть условия неотрицательных переменных ($x_{i, j} > 0$)
  
  Примеры задач линейного программирования вкладываются в общую схему задач математического программирования.
  
  \section*{Общая задача линейного программирования}
  $\begin{cases}
    \min(\max) \sum_{j = 1}^n c_j x_j \\
    \sum_{j = 1}^n a_{i, j} x_j \{=, \leq, \geq\} b_i, i \in \overline{1, m} \\
    x_j \geq 0, j \in \overline{1, m}
  \end{cases}$
  
  \textbf{Целевая функция} - функция, которая минимизируется или максимизируется.
  
  \textbf{Вектор цели} $c = (c_1 \dots с_n)$ определяет целевую функцию.
  
  \textbf{Ограничения} могут быть равенствами или неравенствами.
  
  
  \textbf{Матрица задачи (условий)} $A = (a_{i, j})_{m \times n}$.
  	
  \textbf{Вектор правых частей (ограничений)} $B = (b_1 \dots b_m)$.
  
  \textbf{Условие неотрицательности переменных} $x_j >= 0$.

  \textbf{План задачи} $x = (x_1 \dots x_n)$ - допустимый, если удовлетворяет всем ограничениям.
  
  \textbf{Допустимое множество} $X$ – множество всех допустимых планов задачи.

  $\overline{x} = (\overline{x_1} \dots \overline{x_n})$ - \textbf{Оптимальный план} , если $\forall x \in X: \sum_{j = 1}^n c_j \overline{x_j} \leq \sum_{j = 1}^n c_j x_j$

$\sum_{j = 1}^n c_j \overline{x_j}$ - \textbf{Оптимальное значение задачи}

\textbf{Решение задачи линейного программирования} - найти хотя бы один оптимальный план и вычислить оптимальное значение.

\section*{Частные формы задачи ЛП}
\begin{enumerate}
  \item Планирования производства
  
    $\begin{cases}
      \max \sum_{j = 1}^n c_j x_j \\
      \sum_{j = 1}^n a_{i, j} x_j = b_i, \forall i \\
      x_j \geq 0, \forall j
    \end{cases}$
  \item Каноническая задача
  
    $\begin{cases}
      \min \sum_{j = 1}^n c_j x_j \\
      \sum_{j = 1}^n a_{i, j} x_j = b_i, \forall i \\
      x_j \geq 0, \forall j
    \end{cases}$
  \item Основная задача
    
    $\begin{cases}
      \min \sum_{j = 1}^n c_j x_j \\
      \sum_{j = 1}^n a_{i, j} x_j \leq b_i, \forall i \\
    \end{cases}$
\end{enumerate}

Существуют правила перехода от одной задачи к другой (формы эквивалентны).
\begin{enumerate}
  \item важна с точки зрения приложений
  \item решается алгебраическими методами, приводим задачи к этому виду для решения
  \item важна при рассмотрении теоретических вопросов
\end{enumerate}

\begin{itemize}
\item Матричная запись:

 $X, C, B$ - вектор-столбцы

$\begin{cases}
  \max C^T X \\
  AX \leq B \\
  X \geq 0
\end{cases}$

\item Векторная запись:

$A = \begin{bmatrix}
  a_1 \\ \vdots \\ a_m
\end{bmatrix}$ - набор векторов (строк) 

$\begin{cases}
  \max (C, X) \\
  (a_i, X) \leq b_i, \forall i \\
  X \geq 0
\end{cases}$

\item Запись через столбцы:

$A = (A_1 \dots A_n)$ - набор столбцов

$\begin{cases}
  \max (C, X) \\
  \sum_{j = 1}^n A_j x_j \leq B \\
  X \geq 0
\end{cases}$
\end{itemize}

\subsection*{Правила перехода}
\begin{enumerate}
  \item $\max\limits_{x \in S} f(x) = - \min\limits_{x \in S} (- f(x))$
  
    $x^*$ - точка максимума $f(x)$
  
    $\forall x \in S: f(x^*) \geq f(x)$
  
    $\forall x \in S: - f(x^*) \leq - f(x)$
  
    $x^*$ - точка минимума $- f(x)$
  
  \item $f_i(x) \geq 0 \sim - f_i(x) \leq 0$
  
  \item $(a_i, x) \leq b_i$
  
    Добавим переменную $x_{n + i} = b_i - (a_i, x) \geq 0$
    
    $(a_i, x) \leq b_i \sim
    \begin{cases}
      (a_i, x) + x_{n + i} = b_i \\
      x_{n + i} \geq 0    
    \end{cases}$
    
  \item $(a_i, x) = b_i \sim
    \begin{cases}
      (a_i, x) \leq b_i \\
      - (a_i, x) \leq b_i
    \end{cases}$
    
  \item $a_{i, 1}x_1 + \dots + a_{i, n}x_n = b_i$
  
    Пусть $a_{i, 1} \neq 0$
    
    $x_1 = \frac{1}{a_{i, 1}}(b_i - a_{i, 2}x_2 - \dots - a_{i, n}x_n)$
    
    $x_i \geq 0 \Rightarrow a_{i, 2}x_2 + \dots + a_{i, n}x_n \leq b_i$
    
    Применим метод Жордана-Гаусса. Количество ограничений сократится.
    
  \item $x_j \geq 0 \sim - x_j \leq 0 \sim (a_j, x) \leq b_j, b_j = 0, a_j = -1$
  
  \item $x_j$ - свободная переменная.
  
    Замена $x_j = x'_j - x''_j, x'_j, x''_j \geq 0$
\end{enumerate}

\section*{Геометрическая задача ЛП на плоскости и графический метод решения}
$z = c_1 x_1 + c_2 x_2 \rightarrow \max$

$\begin{cases}
  a_{1, 1}x_1 + a_{1, 2}x_2 \leq b_1 \\
  \vdots \\
  a_{m, 1}x_1 + a_{m, 2}x_2 \leq b_m
\end{cases}$

$x_i \geq 0 \forall i$

\begin{enumerate}
  \item Построение допустимого множества $X$.
  
  $a_{i, 1}x_1 + a_{i, 2}x_2 \leq b_i$
  
  Множество точек, удовлетворяющих неравенству - полуплоскость. Построим прямую $a_{i, 1}x_1 + a_{i, 2}x_2 = b_i$. Вектор нормали $(a_{i, 1}, a_{i, 2})$ направлен в искомую полуплоскость.
  
  Также учитываем условия неотрицательности переменных.
  
  Множество $X$ - пересечение всех полученных полуплоскостей. Возможны три случая:
  \begin{enumerate}
    \item $X$ - многоугольник
    \item $X$ - неограниченное многоугольное множество
    \item $X$ - пустое множество (решений нет)
  \end{enumerate}

  \item Поиск $\max z = \max (c, x)$ с помощью линий уровня.
  
  $(c, x) = z_0$ - начальная прямая. Как правило, $z_0 = 0$.
  
  $c$ - её вектор нормали.
  
  Возьмем $z_1 > z_0$. Прямая $(c, x) = z_1$ параллельно сдвинута в сторону вектора нормали (в задаче на $\min$ сдвигаем в обратную сторону).
  
  Сдвигаем прямую до крайнего положения $(c, x) = f^*$. Множество оптимальных планов $x^*$ - пересечение $X$ и прямой $(c, x) = f^*$.
  
  Если $X$ - неограниченное множество, то возможна (зависит от направления вектора $c$) ситуация, когда $f^* = \infty$.
\end{enumerate}

\section*{Геометрические свойства задачи ЛП в пространстве $R^n$}
\begin{df}{Выпуклая комбинация}

  $x - x_1$ - часть вектора $x_2 - x_1$.
  
  $x - x_1 = \alpha(x_2 - x_1)$.
  
  Значение $\alpha$ задает положение точки $x$ на отрезке $[x_1, x_2]$.
  
  $x = (1 - \alpha)x_1 + \alpha x_2$ - уравнение отрезка $[x_1, x_2]$ при $0 \leq \alpha \leq 1$ (выпуклая комбинация точек $x_1$, $x_2$).
\end{df}

\begin{df}{Выпуклое множество}
  $S$ - выпуклое, если любой отрезок с концами в $S$ целиком содержится в $S$.
  
  $\forall x_1, x_2 \in S: [x_1, x_2] \subset S$
  
  Или:
  
  $\forall x_1, x_2 \in S, \alpha \in [0, 1]: x = (1 - \alpha)x_1 + \alpha x_2 \in S$
\end{df}

\begin{statement}
  Непустое пересечение выпуклых множеств - выпуклое.
\end{statement}
\begin{proof}
  $S = S_1 \cap S_2$

  Возьмем $x_1, x_2 \in S$
  
  $x_1, x_2 \in S_i \Rightarrow [x_1, x_2] \subset S_i \Rightarrow [x_1, x_2] \subset S_1 \cap S_2$
\end{proof}

\begin{df}{Гиперплоскость}
  $x \in R^m$

  $H = \{x|(a, x) = b\}$ - гиперплоскость
  
  $a$ - нормаль гиперплоскости
\end{df}

\begin{statement}
  Гиперплоскость - выпуклое множество.
\end{statement}
\begin{proof}
  $x_1, x_2 \in H, \alpha \in [0, 1]$
  
  $x = (1 - \alpha)x_1 + \alpha x_2$
  
  Скалярное произведение линейно $\Rightarrow$
  
  $(a, x) = (1 - \alpha)(a, x_1) + \alpha (a, x_2) = (1 - \alpha) b + \alpha b = b$
  
  $\Rightarrow x \in H$
\end{proof}

$H$ делит пространство на две полуплоскости:

$H^+ = \{x|(a, x) \geq b\}$

$H^- = \{x|(a, x) < b\}$

\begin{statement}
  $H^+$ и $H^-$ - выпуклые множества.
\end{statement}
\begin{proof}
  $x_1, x_2 \in H^+, \alpha \in [0, 1]$
  
  $x = (1 - \alpha)x_1 + \alpha x_2$
  
  $(a, x) = (1 - \alpha)(a, x_1) + \alpha (a, x_2) = (1 - \alpha)(b + \theta_1) + \alpha (b + \theta_2) = b + (1 - \alpha)\theta_1  + \alpha\theta_2  \geq b$
  
  где $\theta_1, \theta_2 \geq 0$
  
  Для $H^-$ аналогично.
\end{proof}

$H^-_i = \{x|(a_i, x) \leq b_i\}$
  
$X = \bigcap\limits_{i = 1}^m H^-_i$ - допустимое множество основной задачи (многогранник решений)

\begin{statement}
  $X$ - выпуклое множество.
\end{statement}

\begin{df}{Предельная точка}
  $\overline{x} \in X$ - предельная точка, если:
  
  $\exists \{x_k\} \subset X: \forall k \in N: x_k \neq \overline{x}$ и $x_k \xrightarrow[k \to \infty]{} \overline{x}$
\end{df}

Скалярное произведение непрерывно $\Rightarrow (a_i, x_k) \xrightarrow[k \to \infty]{} (a_i, \overline{x}) \leq b_i$

Значит $X$ - замкнутое множество.

\begin{df}{k-грань}
  $x$ - k-грань, если:

  $\exists \{i_1 \dots i_k\} \subset \overline{1, n}$, такое, что:
  
  \begin{enumerate}
    \item $a_{i_l}$ - линейно независимы
    \item $\forall l \in \overline{1, k}: (a_{i_l}, x) = b_{i_l}$
    \item $\forall i \in \overline{1, n}: (a_i, x) \leq b_i$
  \end{enumerate}
\end{df}

\begin{inlinedf}{Вершина}
  $n$-грань
\end{inlinedf}

\begin{mlstatement}{}
  Пусть $X \neq \O$, $r = r(A)$. Тогда $\forall k \in \overline{1, r}: \exists k$-грань
  
  \textit{(Без доказательства)}
\end{mlstatement}

%TODO непонятное утверждение

\begin{df}{Выпуклая комбинация}
  Выпуклая комбинация точек $x_1 \dots x_k$ - $\sum_{i = 1}^k \alpha_i x_i$, где:
  
  $\forall i: \alpha_i = 0$
  
  $\sum_{i = 1}^k \alpha_i = 1$
\end{df}

\begin{statement}
  Выпуклое множество содержит все выпуклые комбинации своих точек.
\end{statement}
\begin{proof}
  Пусть $S$ - выпуклое множество, $x_1 \dots x_m \in S$
  
  $x = \sum_{i = 1}^m \alpha_i x_i$
  
  Докажем, что $x \in S$ индукцией по числу точек $m$:
  
  \textit{Б.И.} $k = 2$
  
  Множество значений $x$ - отрезок. Из определения выпуклого множества $x \in S$.
  
  \textit{Ш.И.}
  
  Рассмотрим три случая:
  \begin{enumerate}
    \item $\alpha_m = 0$
    
    Количество слагаемых - $m - 1$. Выполнено П.И.
    \item $\alpha_m = 1$
    
    $\sum_{i = 1}^m \alpha_i = 1 \Rightarrow$ одно слагаемое, очевидно.
    \item $0 < \alpha_m < 1$
    
    $x = (1 - \alpha_m) \sum_{i = 1}^{m - 1} \frac{\alpha_i}{1 - \alpha_m} x_i + \alpha_m x_m = (1 - \alpha_m)y + \alpha_m x_m$
    
    $\sum_{i = 1}^{m - 1} \alpha_i = 1 - \alpha_m \Rightarrow \sum_{i = 1}^{m - 1} \frac{\alpha_i}{1 - \alpha_m} = 1 \Rightarrow y$ - выпуклая комбинация $m - 1$ точек и по П.И. $y \in S$
    
    $\Rightarrow x$ - выпуклая комбинация 2-х точек из $S \Rightarrow x \in S$
  \end{enumerate}
\end{proof}

\begin{df}{Выпуклая оболочка}
  Выпуклая оболочка множества $M$ - множество всех выпуклых комбинаций точек из $M$.
\end{df}

Свойства $coM$:
\begin{enumerate}
  \item $coM$ - выпуклое множество.
  
    $x_1, x_2 \in coM$
    
    $x_1 = \sum_{i = 1}^k \alpha_i x_i$
    
    $x_2 = \sum_{j = 1}^m \beta_j x_j^*$
    
    $x = \lambda x_1 + (1 - \lambda)x_2$
    
    Легко проверить, что $x$ - выпуклая комбинация $x_1 \dots x_k, x_1^* \dots x_m^*$.
  \item $M \subset coM$
  
  Любая точка множества - выпуклая комбинация из одного слагаемого.
  \item $\forall S$ - выпуклое множество: $M \subset S \Rightarrow coM \subset S$
  
    $S$ - выпуклое, значит, оно содержит все выпуклые комбинации своих точек.
    
    $M \subset S \Rightarrow S$ содержит все выпуклые комбинации точек из $M \Rightarrow coM \subset S$
\end{enumerate}

\begin{example}
  Треугольник - выпуклая оболочка его трех вершин.
  
  \begin{proof}
    $x_1, x_2, x_3$ - вершины.
    
    $y = \alpha x_1 + (1 - \alpha)x_2$ - отрезок $[x_1, x_2]$
    
    $x = \beta y + (1 - \beta)x_3$ - треугольник
    
    $x = \beta \alpha x_1 + \beta(1 - \alpha)x_2 + (1 - \beta)x_3$
    
    Сумма коэффициентов равна единице $\Rightarrow x$ - выпуклая комбинация.
  \end{proof}
  
  Многоугольник - выпуклая оболочка своих вершин.
\end{example}

\begin{df}{Крайняя точка}
  $\overline{x} \in M$ - крайняя (угловая) точка, если ее нельзя представить в виде нетривиальной ($\alpha > 0$) выпуклой комбинации двух различных точек из $M$.
\end{df}

\begin{example}
  Крайние точки многоугольника - его вершины. У сферы все точки являются крайними.
\end{example}

\begin{theorem}
  Любое непустое выпуклое замкнутое ограниченное множество из $R^n$ можно представить как выпуклую оболочку своих крайних точек.
\end{theorem}
%TODO доказательство теоремы

\begin{inlinedf}{Выпуклый многогранник}
  выпуклая оболочка конечного числа точек.
\end{inlinedf}

Свойства:
\begin{enumerate}
  \item выпуклое множество
  \item ограниченное
  \item замкнутое
  
  \begin{proof}
    $M$ - выпуклый многогранник.  
  
    $\{x_k\} \subset M$
    
    $x_k \xrightarrow[k \to \infty]{} \overline{x}$
    
    $x_k = \sum_{i = 1}^m \alpha_i^k x^i$
    
    $\forall k: \sum_{i = 1}^m \alpha_i^k = 1$
    
    Последовательность $\{\alpha_i^k\}_{k = 0}^\infty$ ограничена. Выделим сходящуюся подпоследовательность для каждого $i$:
    
    $i = 1: \exists \{x_{k_j^1}\} \subset \{x_k\}: \alpha_1^{k_j^1} \xrightarrow[j \to \infty]{} \overline{\alpha_1}$
    
    $i = 2: \exists \{x_{k_j^2}\} \subset \{x_{k_j^1}\}: \alpha_2^{k_j^2} \xrightarrow[j \to \infty]{} \overline{\alpha_2}$

    $\vdots$
    
    $i = m: \exists \{x_{k_j^m}\} \subset \{x_{k_j^{m - 1}}\}: \alpha_m^{k_j^m} \xrightarrow[j \to \infty]{} \overline{\alpha_m}$
    
    Имеем $x_{k_j^m} \xrightarrow[j \to \infty]{} \sum_{i = 1}^m \overline{\alpha_i} x^i$
    
    При этом $\sum_{i = 1}^m \overline{\alpha_i} = 1$
    
    Но $x_k \xrightarrow[k \to \infty]{} \overline{x}$, а $x_{k_j^m}$ - подпоследовательность $x_k$, значит $\overline{x} =  \sum_{i = 1}^m \overline{\alpha_i} x^i$ - выпуклая комбинация точек из $M$. Следовательно, $\overline{x} \in M$.
  \end{proof}
\end{enumerate}

Из теоремы следует, что выпуклый многогранник - выпуклая оболочка своих \textbf{крайних} точек.

$X = {x | Ax \leq B}$ - допустимое множество задачи ЛП.

$X$ - выпуклое и замкнутое.

\begin{enumerate}
  \item $X$ - ограничено.
  
  По теореме $X$ можно представить как выпуклую оболочку своих крайних точек (вершин)
  
  Количество вершин конечно, а точнее, не превосходит $C_m^n$, где $n$ - размерность задачи, $m$ - количество ограничений.
  
  Значит $X$ - выпуклый многогранник.
  \item $X$ - неограниченно.
  
  Пусть $p_1 \dots p_s$ - вершины $X$.
  
  У $X$ есть два неограниченных ребра. Возьмем их направляющие векторы $S_1, S_2$.
  
  Тогда $x = \sum_{i = 1}^s \alpha_i p_i + \beta_1 S_1 + \beta_2 S_2$.
  %TODO неочевидно
\end{enumerate}

Если $X$ - ограничено и непусто, то задача ЛП имеет решение.

\textbf{Признак разрешимости:} Если целевая функция задачи ЛП на $\max$ ($\min$) ограничена сверху (снизу) на допустимом множестве, то задача разрешима.

\begin{statement}
  Если задача ЛП разрешима, то среди решений есть хотя бы одна вершина допустимого множества.
\end{statement}
\begin{proof}
  $\begin{cases}
    \max (c, x) \\
    (a_i, x) \leq b_i, i \in \overline{1, m} \\
    x_i \geq 0, i \in \overline{1, m}
  \end{cases}$ - задача ЛП
  
  $x^*$ - решение
  
  %TODO почему ограничено?
  $X$ - ограничено $\Rightarrow x^* = \sum_{i = 1}^N \alpha_i p_i$, где $p_1 \dots p_N$ - вершины $X$.
  
  $f^* = (c, x^*) = \sum_{i = 1}^N \alpha_i (c, p_i)$
  
  Возьмем $p = \max\limits_{1 \leq i \leq N} p_i$
  
  Тогда $f^* \leq (c, p) \sum_{i = 1}^N \alpha_i = (c, p)$
  
  Но $f^*$ - оптимальное значение функции $\Rightarrow \forall x \in X: f^* \geq (c, x)$
  
  Отсюда, $(c, x^*) = (c, p) \Rightarrow x^* = p \Rightarrow x^*$ - вершина. 
\end{proof} 

\begin{statement}
  Любая выпуклая комбинация решений является решением.
\end{statement}
\begin{proof}
  $p_1 \dots p_k$ - решения
  
  $\forall i \in \overline{1, k}: f^* = (c, p_i)$
  
  $p = \sum_{i = 1}^k \alpha_i p_i$
  
  $(c, p) = \sum_{i = 1}^k \alpha_i (c, p_i) = \sum_{i = 1}^k \alpha_i f^* = f^*$
\end{proof}

\subsection*{Опорный план}
Определен для канонической задачи ЛП:

$\begin{cases}
  \min (c, x) \\
  \sum_{j = 1}^n A_j x_j = B \\
  x \geq 0
\end{cases}$

\begin{df}{Опорный план}
  План задачи $\overline{x} = (\overline{x_1} \dots \overline{x_n})$ является опорным, если столбцы $\{A_j | \overline{x_j} > 0\}$ линейно независимы.
  
  Если столбцов столько же, сколько и ограничений, то план - невырожденный.
\end{df}

\begin{statement}
  $\overline{x}$ - опорный план $\Longleftrightarrow \overline{x}$ - крайняя точка. 
\end{statement}
\begin{proof}
\underline{Необходимость}

Пусть $\overline{x}$ - опорный план.

Предположим от противного: $\overline{x}$ - не крайняя точка.

Тогда $\exists \alpha_1, \alpha_2 > 0: \exists x', x'' \in X, x' \neq x'': \overline{x} = \alpha_1 x' + \alpha_2 x'', \alpha_1 + \alpha_2 = 1$

$x'$ и $x''$ удовлетворяют ограничениям задачи ЛП:

$\begin{cases}
  \sum_{j = 1}^n A_j x'_j = B \\
  \sum_{j = 1}^n A_j x''_j = B
\end{cases}$

Пусть $I = \{j | \overline{x_j} > 0\}$ - множество индексов. Тогда $\forall j \notin I: x'_j = x''_j = 0$. Вычитая одно равенство из другого получаем:

$\sum_{j \in I} A_j (x'_j - x''_j) = 0$

$A_j$ - линейно независимы, значит $\forall j \in I: x'_j = x''_j$. При этом $\forall j \notin I: x'_j = x''_j = 0$. Отсюда, $x' = x''$, противоречие.\\~\\

\underline{Достаточность}

Пусть $\overline{x}$ - крайняя точка, $I = \{j | \overline{x_j} > 0\}$ - множество индексов при ненулевых компонентах.

$\overline{x}$ удовлетворяет ограничениям задачи ЛП: $\sum_{j \in I} A_j \overline{x_j} = B$

Предположим от противного: $\overline{x}$ - не опорный план. Тогда $\{A_j\}_{j \in I}$ - линейно зависимы, то есть, существует нетривиальная линейная комбинация $d_1 \dots d_k$, такая, что:

$\sum_{j \in I} d_j A_j = 0$

Возьмем $\epsilon > 0$. Умножим обе части равенства на $\pm \epsilon$ и сложим с ограничением задачи ЛП. Получим:

$\sum_{j \in I} A_j (\overline{x_j} \pm \epsilon d_j) = B$

Выберем $\epsilon$ таким, что $\forall j \in I: x_j \pm \epsilon d_j \geq 0$

Пусть вектора $x', x''$ имеют координаты:

$\forall j \in I: \begin{cases}
  x'_j = \overline{x_j} - \epsilon d_j \\
  x''_j = \overline{x_j} + \epsilon d_j
\end{cases}$

$\forall j \notin I: x'_j = x''_j = 0$

$x'$ и $x''$ имеют неотрицательные координаты и удовлетворяют ограничениям задачи лп, значит, являются планами задачи. Кроме того, $x' \neq x''$ и $\overline{x} = \frac{1}{2}x' + \frac{1}{2}x''$. Следовательно, $\overline{x}$ - выпуклая комбинация различных точек из $X$, а значит, не является крайней, противоречие.
\end{proof}

\section*{Симплекс метод}
Условия применимости:
\begin{enumerate}
  \item Задача - в каноническом виде
  \item Число уравнений строго меньше числа неизвестных
  \item Все свободные члены $b_i \geq 0$
  \item $r(A) = m$. Матрица $A$ содержит единичную $m$-мерную подматрицу
  
  Без ограничения общности будем считать, что
  $\forall k \in \overline{1, m}: A_{k, j} = 
  \begin{cases}
    0, j \neq k \\
    1, j = k
  \end{cases}$
  
  $x_1 \dots x_m$ - базисные компоненты, остальные - небазисные
\end{enumerate}

Начальный опорный план $x^1 = (x^1_1 \dots x^1_m, 0 \dots 0),\ \ x^1_i = b_i$

Выразим столбцы матрицы $A$ через столбцы при базисных компонентах:

$\forall j \in \overline{1, n}: A_j = \chi_{1, j}A_1 + \dots + \chi_{m, j}A_m$

Такое разложение существует, т.к. $A_1 \dots A_m$ - линейно независимы, а значит являются базисом $m$-мерного пространства столбцов.

$X = \{\chi_{i, j}\}_{m \times n}$ - матрица коэффициентов разложения.

Разложение в матричном виде: $A = [A_1 \dots A_m]X$ \\~\\

Пусть $A_{m + 1} = \chi_{1, m+1}A_1 \dots \chi_{m, m+1}A_m$ - разложение некоторого небазисного столбца $A_{m + 1}$. Пусть $\chi_{1, m+1} > 0$.

Положим $\theta > 0$ - некоторый неопределенный множитель.

Умножим равенство выше на $\theta$ и вычтем из ограничений задачи. Получим:

$A_1(x^1_1 - \theta \chi_{1, m+1}) + \dots + A_m(x^1_m - \theta \chi_{m, m+1}) + A_{m+1}\theta = b$

Пусть $x_\theta = (x^1_1 - \theta \chi_{1, m+1}, \dots, x^1_m - \theta \chi_{m, m+1}, \theta, 0, \dots, 0)$

Из полученного равенства следует, что $x_\theta$ удовлетворяет ограничениям задачи. Для того, чтобы $x_\theta$ был допустимым планом нужно, чтобы $\forall i: x^1_i - \theta \chi_{i, m+1} \geq 0$

Если $\chi_{i, m+1} \leq 0$, то верно для любого $\theta > 0$. Если $\chi_{i, m+1} > 0$, то верно при $0 < \theta \leq \frac{x^1_i}{\chi_{i, m+1}}$.

Возьмем $\theta \leq \theta_0 = \min\limits_{\chi_{i, m+1} > 0} \frac{x^1_i}{\chi_{i, m+1}}$.

Пусть $\min$ достигается при $i = 1$. Тогда, подставляя $\theta = \theta_0$, получим:

$x^2 = (0, x^2_2, \dots, x^2_m, \theta_0, 0, \dots, 0)$ - новый допустимый план задачи.

Докажем, что $x^2$ является опорным.

От противного: пусть столбцы $A_2 \dots A_{m+1}$ - линейно зависимы. Значит:

$\exists \alpha_2 \dots \alpha_{m+1}$ - нетривиальная л.к.: $\alpha_2 A_2 + \dots + \alpha_{m+1} A_{m+1} = 0$

$A_1 \dots A_m$ - лин. независимы $\Rightarrow A_2 \dots A_m$ - лин. независимы $\Rightarrow \alpha_{m + 1} \neq 0$

Отсюда $A_{m+1} = \sum_{i = 2}^m \beta_i A_i$,\ \ $\beta_i = \frac{\alpha_1}{\alpha_{m+1}}$

Вычтем это равенство из разложения $A_{m+1}$ через базисные столбцы:

$0 = \chi_{1, m+1}A_1 + (\chi_{2, m+1} - \beta_2)A_2 + \dots + (\chi_{m, m+1} - \beta_m)A_m$

$A_1 \dots A_m$ - лин. независимы, значит коэффициенты равны нулю. В частности:

$\chi_{1, m+1} = 0$

Но мы ранее положили, что $\chi_{1, m+1} > 0$, противоречие. \\~\\

В итоге мы перешли к новому опорному плану. Теперь сравним значения целевой функции.

Возьмем разложение $A_j = \chi_{1, j}A_1 + \dots + \chi_{m, j}A_m$ и подставим вместо $A_1 \dots A_m$ коэффициенты целевого вектора $c$ с базисными компонентами:

$z_j = \sum_{i = 1}^m \chi_{i, j} c_{B_i}$

Или в матричной записи:

$z^T = c_B X$

Введем оценки опорного плана $\Delta_j = z_j - c_j$. Понятно, что $\forall j \in \overline{1, m}: \Delta_j = 0$

$z^2_0 = (c, x^2) = c_2 x^2_2 + \dots + c_m x^2_m + c_{m+1}x^2_{m+1} = c_2(x^1_2 - \theta_0 \chi_{2, m+1}) + \dots + c_m(x^1_m - \theta_0 \chi_{m, m+1}) + c_{m+1}\theta_0$

Прибавим $c_1(x^1_1 - \theta_0 \chi_{1, m+1}) = 0$:

$F^* = \dots + c_1(x^1_1 - \theta_0 \chi_{1, m+1}) = c_1 x^1_1 + \dots + c_m x^1_m - \theta_0(c_1 \chi_{1, m+1} + \dots + c_m \chi_{m, m+1}) + c_{m+1} \theta_0 = (c, x^1) - \theta_0 (z_{m+1} - c_{m+1}) = z^1_0  - \theta_0 \Delta_{m+1}$

Как итог: $z^2_0 = z^1_0 - \theta_0 \Delta_{m+1}$ \\~\\

\begin{statement}{О сходимости симплекс метода}
  \begin{enumerate}
    \item Пусть $\exists j: \Delta_j > 0$ и $\exists i: \chi_{i, j} > 0$. Тогда $z^2_0 < z^1_0$
    \item Пусть $\exists j: \Delta_j > 0$ и $\forall i: \chi_{i, j} \leq 0$. Тогда $z_0 = - \infty$ (оптимального плана не существует) %TODO но это не точно
    \item Пусть $\forall j: \Delta_j \leq 0$. Тогда $x^1$ - оптимальный план
  \end{enumerate}
\end{statement}
\begin{proof}
  \begin{enumerate}
    \item Мы это уже доказали, взяв б.о.о $j = m + 1$ и $i = 1$.
    \item б.о.о. $j = m + 1$
      
      Мы доказывали, что $x_\theta = (x^1_1 - \theta \chi_{1, m+1}, \dots, x^1_m - \theta \chi_{m, m+1}, \theta, 0, \dots, 0)$ удовлетворяет ограничениям задачи. Если $\forall i: \chi_{i, j} \leq 0$, то $\forall \theta > 0: x_\theta$ является допустимым планом задачи.
      
      $(c, x_\theta) = z^1_0 - \theta \Delta_{m+1} \xrightarrow[\theta \to \infty]{} - \infty$
      
      Целевую функцию можно уменьшать сколь угодно, значит оптимального плана не существует.
    \item
  \end{enumerate}
\end{proof}

\subsection*{Алгоритм симплекс метода}
\textbf{Шаг 0:}

$B^0 = A_{i_1} \dots A_{i_m}$ - базисные столбцы. $A_{i_k}$ - единица на $k$-ой позиции, все остальные - нули.

$I^0_B = \{i_1 \dots i_m\}$ - номера базисных векторов.

$J^0_N = \overline{1, n} \setminus I^0_B$

$x^0$ - начальный план. $x^0_i = \begin{cases}
  b_k, i = i_k \in I^0_B \\
  0, i \in J^0_N
\end{cases}$

\textbf{Шаг k:}
\begin{enumerate}
  \item Разлагаем $A_j$ по векторам из $B^k$. Получаем матрицу разложений $X^k$
  \item Находим $z^k_j$. Матрица $(z^k)^T = c_{B_k} X^k$
  \item Находим оценки $\Delta^k_j = z^k_j - c_j$
  \item Если $\forall j \in J^k_N: \Delta^k_j \leq 0$, то стоп. $x^k$ является оптимальным планом задачи.
  \item Введем множество $J^N_> = \{j|\Delta^k_j \geq 0\}$
  \item Если $\exists j \in J^N_>: \forall i: \chi^k_{i, j} \leq 0$, то стоп. Оптимального плана не существует.
  \item Введем $p$, равный такому $j \in J^N_>$, что $\Delta^k_j$ максимально. Это будет номер столбца, который мы добавим в базис.
  \item Ведем $l$ такой, что $\theta_0 = \frac{x^k_l}{\chi^k_{l, p}}$ (достигается минимум). Это будет номер столбца, который мы удалим из базиса.
  \item $B^{k + 1} = B^k \cup \{A_p\} \setminus \{A_l\}$
  
  $\forall i \in I^{k+1}_B: x^{k+1}_i = x^k_i - \theta_0 \chi^k_{i, p}$
\end{enumerate}

\textit{Примечание:} возможно зацикливание симплекс метода, если $\theta_0 = \min\limits_{\chi_{i, p} > 0} \frac{x_i}{\chi_{i, p}}$ - минимум достигается при нескольких $i$. Для избежания этого необходимо в таких ситуациях выбирать наименьший из всех $i$. 

\subsection*{Симплекс таблица}
%TODO https://youtu.be/XSzh_PPlxk4

Выведем новое разложение матрицы $A$ через базисные столбцы.

Старое разложение:

$A_0 = x^1_1 A_1 + \dots + x^1_m A_m$

$A_p = \chi_{1, p} A_1 + \dots + \chi_{l, p} A_l + \dots + \chi_{m, p} A_m$

$\forall j: A_j = \chi_{1, j} A_1 + \dots + \chi_{l, j} A_l + \dots + \chi_{m, j} A_m$

Из разложения $A_p$ получаем:

$A_l = \frac{1}{\chi_{l, p}}(A_p -  \chi_{1, p}A_1 - \dots - \chi_{m, p}A_m)$

Подставляем $A_l$ в разложение $A_0$:

$A_0 = A_1(x^1_1 - \frac{x^1_l}{\chi_{l, p}} \chi_{1, p}) + \dots + A_p \frac{x^1_l}{\chi_{l, p}} + \dots + A_m(x^1_m - \frac{x^1_l}{\chi_{l, p}} \chi_{m, p})$

Получаем:

$\begin{cases}
  \overline{x_i} = x_i - \frac{x_l}{\chi_{l, p}} \chi_{i, p}, i \neq l \\
  \overline{x_l} = \frac{x_l}{\chi_{l, p}}
\end{cases}$

- формулы для пересчета столбца $A_0$

Аналогично выводим формулы для пересчета остальных столбцов:

$\begin{cases}
  \overline{\chi_{i, j}} = \chi_{i, j} - \frac{\chi_{l, j}}{\chi_{l, p}} \chi_{i, p}, i \neq l \\
  \overline{\chi_{l, j}} = \frac{\chi_{l, i}}{\chi_{l, p}}
\end{cases}$

Это - формулы прямоугольного метода Жордана-Гаусса.

\textit{Примечание:} если на последнем шаге симплекс метода $\exists j \notin I_B: \Delta_j = 0$, то мы можем изменить базис, а значит, и опорный план, не изменив при этом значение целевой функции. Это значит, что решение задачи не единственно.

\subsection*{Метод искусственного базиса}
Дана задача ЛП:

$\begin{cases}
  \max (c, x) \\
  \forall i: (a_i, x) \{=, \leq, \geq\} b_i \\
  x \geq 0
\end{cases}$

Если ограничения со знаком $\leq$ и $\forall i: b_i \geq 0$, можно перейти к канонической задаче:

$(a_i, x) + x_{n + i} = b_i$

$x_{n + i} \geq 0$ - дополнительные переменные.

Существует единичная подматрица матрицы ограничений, значит, можно применить симплекс метод.

Если ограничения со знаком $\geq$, то при переходе к канонической задаче не всегда можно найти единичную подматрицу. В таком случае ставим расширенную задачу:

$\begin{cases}
  \min (c, x) + M \sum_{i = 1}^m x_{n + i} \\
  \forall i: (a_i, x) + x_{n + i} = b_i \\
  x_1 \dots x_{n + m} \geq 0
\end{cases}$

Здесь $M$ - некоторое большое число, из-за которого все решения c ненулевыми $x_{n + i}$ являются неоптимальными.

$x_{n + i}$ - фиктивные (искусственные) переменные.

К задаче применим симплекс метод:

$\overline{X} = (\overline{x_1}, \dots, \overline{x_{n + m}})$ - оптимальный план.

$\overline{z} = Z(\overline{X})$

\begin{statement}
  Если $\forall j \in \overline{n+1, n+m}: \overline{x_j} = 0$, то $\overline{x} = (\overline{x_1}, \dots, \overline{x_n})$ является решением исходной задачи.
\end{statement}
\begin{proof}
  От противного: $\overline{x}$ не является решением. Тогда $\exists x^*: z(x^*) < z(\overline{x})$
  
  $X^* = (x^*_1, \dots, x^*_n, 0, \dots, 0)$ - допустимый план расширенной задачи.
  
  $Z(X^*) = z(x^*) < z(\overline{x}) = Z(\overline{X})$
  
  Но $\overline{X}$ - оптимальный план расширенной задачи. Противоречие.\\~\\
  
  Утверждение верно и в \underline{обратную сторону}:
  
  От противного: $\exists i: \overline{x_{n + i}} > 0$
  
  Положим, что допустимое множество исходной задачи непусто ($X \neq \O$)
  
  Тогда $\exists X' = (x_1', \dots, x_n', 0, \dots, 0)$ - некоторый план расширенной задачи
  
  $Z(\overline{X}) \geq (c, \overline{x}) + M \overline{x_{n + i}} > (c, x') = Z(X')$
  
  Но $\overline{X}$ - оптимален. Противоречие.
\end{proof}

У таблицы расширенной задачи строка для оценок ($\Delta_i$) разбивается на две. Во вторую строку записывается коэффициент при числе $M$. Вторая строка имеет приоритет при выборе вектора, добавляемого в базис.

\end{sloppypar}
\end{document}
